\documentclass[12pt,a4paper]{report}

%--------------------------------------
\usepackage[T1]{fontenc} %Not needed by LuaLaTeX or XeLaTeX
%--------------------------------------

%Portuguese-specific commands
%--------------------------------------
%\usepackage[portuguese]{babel}
%--------------------------------------

%Hyphenation rules
%--------------------------------------
\usepackage{hyphenat}
\hyphenation{mate-mática recu-perar}
%--------------------------------------

%Espaçamento e outras cenas
%--------------------------------------
\usepackage{setspace}
\usepackage{anyfontsize}
\usepackage{indentfirst}
\usepackage{parskip}
\usepackage{titlesec} % Chapter
\usepackage{geometry} % Margens
\usepackage{amsmath} % Matemática
\usepackage{graphicx}
\usepackage{wrapfig}
\usepackage{subcaption}
\usepackage{caption}
\usepackage[hidelinks]{hyperref}
\usepackage{xpatch}
\usepackage{etoolbox}
\usepackage{titletoc}
\usepackage{times}
\usepackage{fancyhdr}
\usepackage{lastpage}
\usepackage{xcolor}
\usepackage{listings}
\usepackage{tabularx}
\usepackage{array}
%------------------------------

%Margens
\geometry{
    a4paper,
    right = 3cm,
    left = 2.5cm,
    top = 2.5cm,
    bottom = 2.5cm,
}

\renewcommand{\contentsname}{Índice}
\newcommand{\code}{\texttt\textbf}

\setcounter{secnumdepth}{0} % Não enumerar secções

\setstretch{1.5} % espaçamento

\titleformat{\chapter}{\fontfamily{ptm}\selectfont\centering\huge\titlerule[1.5pt]\vspace{-9pt}}{}{0pt}{\Huge}[{\titlerule[1.5pt]}]

\titleformat{\section}{\bfseries\fontfamily{ptm}\selectfont}{}{0pt}{\large}[{\titlerule[1pt]}\vspace{10pt}]

\titleformat{\subsection}{\bfseries\fontfamily{ptm}\selectfont}{}{0pt}{\large}[\vspace{2pt}]

\titlespacing{\chapter}{0pt}{-30pt}{30pt}

\titlecontents{section}[40pt]{\vskip3pt\bfseries}{\thecontentslabel\quad}{}{~~\normalfont\dotfill\bfseries\contentspage}[]



\begin{document}

\chapter{Introdução}

Se no primeiro relatório explicámos com detalhe cada uma das diferentes estruturas de dados implementas e apresentámos as estratégias que adotámos para responder o mais rapidamente possível às análises que eram pretendidas, neste relatório iremos focar a nossa apreciação num ponto de vista mais ao nível do \textit{hardware} e tempos de execução, algo que certamente até os menos entendidos na área da informática serão capazes de compreender.

Deste modo, pretendemos apresentar e justificar os diversos tempos de resposta que o nosso programa detém, quer na transferência dos dados para a memória, quer na análise de cada uma das \textit{queries}. Além disso, como a quantidade de recursos requeridos por um programa parece-nos ser um parâmetro de extrema relevância, decidimos por bem referir alguns dados relativos à memória.

Por fim, como naturalmente algumas \textit{queries} levam significativamente mais tempo do que outras a serem executadas, procurámos justificar tais resultados a partir de uma análise assimptótica que levava em conta diversos parâmetros, pois desta forma julgamos estar a ser o mais rigorosos possível.

\chapter{Testes de Desempenho}

O tempo é um dos parâmetros mais relevantes no que à execução de programas diz respeito, posto isto, decidimos analisar o tempo de execução do nosso projeto não só no global, mas também em diferentes etapas da execução em si, entre elas a recolha dos dados e o tempo que cada \textit{query} leva a ser analisada.

Deste modo, para realizar uma análise da forma mais correta possível, executámos o dito programa em computadores distintos, pois assim sempre podemos ter uma ideia de qual o \textit{hardware} que melhor se adequa ao código gerado.


\section{Tratamento estatístico}

\begin{itemize}
    
    \item Realização de cinco execuções em cada um dos computadores (ligados à corrente).
    
    \item Eliminação de \textit{outliers} resultantes de execuções nas quais o sistema operativo estava livre/sobrelotado.

    \item Cada um dos tempos resulta de uma média ponderada sem a consideração dos \textit{outliers}. 
\end{itemize}


\section{Ambientes de Execução}

{
\setlength\arrayrulewidth{1pt}

\begin{center}
    \begin{tabular}{ |p{4.73cm}|p{4.73cm}|p{4.73cm}|  }
        \hline
        \multicolumn{3}{|c|}{\textbf{\texttt{Computador A}}} \\
        \hline
        \centering\textbf{\texttt{Processador}} & \hfil \textbf{\texttt{Memória}} & \hfil \textbf{\texttt{Sistema Operativo}} \\
        \hline
        \centering\textbf{\texttt{Intel i5-10300H}} & \hfil \texttt{\textbf{8GB - 2600MHz}} & \hfil \texttt{\textbf{Ubuntu}}\\
        \hline
    \end{tabular}
\end{center}


\vspace{10pt}

\begin{center}
    \begin{tabular}{ |p{4.73cm}|p{4.73cm}|p{4.73cm}|  }
        \hline
        \multicolumn{3}{|c|}{\textbf{\texttt{Computador B}}} \\
        \hline
        \centering\textbf{\texttt{Processador}} & \hfil \textbf{\texttt{Memória}} & \hfil \textbf{\texttt{Sistema Operativo}} \\
        \hline
        \centering\textbf{\texttt{Intel i5-1135G7}} & \hfil \texttt{\textbf{16GB - 2700MHz}} & \hfil \texttt{\textbf{Ubuntu}}\\
        \hline
    \end{tabular}
\end{center}
}


\newpage

\section{Desempenho das Queries - Dataset Regular}

{
\setlength\arrayrulewidth{1pt}
\begin{tabularx}{\textwidth} { 
  | >{\centering\arraybackslash}X
  | >{\centering\arraybackslash}X
  | >{\centering\arraybackslash}X | }
 \hline
 \texttt{\textbf{Query}} & \texttt{\textbf{Computador A (s)}} & \texttt{\textbf{Computador B (s)}}  \\
 \hline
 \texttt{\textbf{1 SaCruz110}} & \texttt{\textbf{0.000073}} & \texttt{\textbf{0.000072}}  \\
 \hline
 \texttt{\textbf{1 ClarPacheco48}} & \texttt{\textbf{0.000053}} & \texttt{\textbf{0.000021}} \\
 \hline
 \texttt{\textbf{1 GabriJesus}} & \texttt{\textbf{0.000013}} & \texttt{\textbf{0.000005}} \\
 \hline
 \texttt{\textbf{1 000000004780}} & \texttt{\textbf{0.000041}} & \texttt{\textbf{0.000018}} \\
 \hline
 \texttt{\textbf{1 000000007141}} & \texttt{\textbf{0.000027}} & \texttt{\textbf{0.000018}} \\
 \hline
 \texttt{\textbf{1 000000003123}} & \texttt{\textbf{0.000028}} & \texttt{\textbf{0.000005}} \\
 \hline
 \texttt{\textbf{2 10}} & \texttt{\textbf{0.177266}} & \texttt{\textbf{0.207539}} \\
 \hline
 \texttt{\textbf{2 50}} & \texttt{\textbf{0.180220}} & \texttt{\textbf{0.205983}} \\
 \hline
 \texttt{\textbf{3 10}} & \texttt{\textbf{0.236767}} & \texttt{\textbf{0.264780}} \\
 \hline
 \texttt{\textbf{4 Braga}} & \texttt{\textbf{0.013727}} & \texttt{\textbf{0.011439}} \\
 \hline
 \texttt{\textbf{4 Porto}} & \texttt{\textbf{0.011629}} & \texttt{\textbf{0.010332}} \\
 \hline
 \texttt{\textbf{4 Lisboa}} & \texttt{\textbf{0.011562}} & \texttt{\textbf{0.010240}} \\
 \hline
 \texttt{\textbf{5 01/01/2020 01/01/2021}} & \vspace{-5pt}\texttt{\textbf{0.054139}} & \vspace{-5pt}\texttt{\textbf{0.053338}} \\
 \hline
 \texttt{\textbf{5 01/01/2021 01/01/2022}} & \vspace{-5pt}\texttt{\textbf{0.058550}} & \vspace{-5pt}\texttt{\textbf{0.058892}} \\
 \hline
 \texttt{\textbf{6 Braga 01/01/2020 01/01/2021}} & \vspace{-5pt}\texttt{\textbf{0.017424}} & \vspace{-5pt}\texttt{\textbf{0.014922}} \\
 \hline
 \texttt{\textbf{6 Porto 01/01/2021 01/02/2021}} & \vspace{-5pt}\texttt{\textbf{0.017406}} & \vspace{-5pt}\texttt{\textbf{0.014893}} \\
 \hline
 \texttt{\textbf{7 10 Braga}} & \texttt{\textbf{0.038273}} & \texttt{\textbf{0.037780}} \\
 \hline
 \texttt{\textbf{8 M 12}} & \texttt{\textbf{0.013602}} & \texttt{\textbf{0.015746}} \\
 \hline
 \texttt{\textbf{8 F 12}} & \texttt{\textbf{0.014444}} & \texttt{\textbf{0.016966}} \\
 \hline
 \texttt{\textbf{9 24/12/2021 25/12/2021}} & \vspace{-5pt}\texttt{\textbf{0.042344}} & \vspace{-5pt}\texttt{\textbf{0.038273}} \\
 \hline
 \texttt{\textbf{9 9 01/01/2012 01/01/2013}} & \vspace{-5pt}\texttt{\textbf{0.041159}} & \vspace{-5pt}\texttt{\textbf{0.036926}} \\
 \hline
 \end{tabularx}
}


\newpage

\section{Desempenho das Queries - Dataset Estendido}

{
\setlength\arrayrulewidth{1pt}
\begin{tabularx}{\textwidth} { 
  | >{\centering\arraybackslash}X
  | >{\centering\arraybackslash}X
  | >{\centering\arraybackslash}X | }
 \hline
 \texttt{\textbf{Query}} & \texttt{\textbf{Computador A (s)}} & \texttt{\textbf{Computador B (s)}} \\
 \hline
 \texttt{\textbf{1 EmanSimões9}} & \texttt{\textbf{0.000079}} & \texttt{\textbf{0.000084}} \\
 \hline
 \texttt{\textbf{1 GuilAraújo529}} & \texttt{\textbf{0.000018}} & \texttt{\textbf{}} \\
 \hline
 \texttt{\textbf{1 MateuPereira}} & \texttt{\textbf{0.000018}} & \texttt{\textbf{}} \\
 \hline
 \texttt{\textbf{1 000000063182}} & \texttt{\textbf{0.000023}} & \texttt{\textbf{}} \\
 \hline
 \texttt{\textbf{1 000000030891}} & \texttt{\textbf{0.000022}} & \texttt{\textbf{}} \\
 \hline
 \texttt{\textbf{1 000000095244}} & \texttt{\textbf{0.000020}} & \texttt{\textbf{}} \\
 \hline
 \texttt{\textbf{2 10}} & \texttt{\textbf{2.024240}} & \texttt{\textbf{}} \\
 \hline
 \texttt{\textbf{2 50}} & \texttt{\textbf{2.027261}} & \texttt{\textbf{}} \\
 \hline
 \texttt{\textbf{3 10}} & \texttt{\textbf{2.708971}} & \texttt{\textbf{}} \\
 \hline
 \texttt{\textbf{4 Braga}} & \texttt{\textbf{0.234393}} & \texttt{\textbf{}} \\
 \hline
 \texttt{\textbf{4 Porto}} & \texttt{\textbf{0.234568}} & \texttt{\textbf{}} \\
 \hline
 \texttt{\textbf{4 Lisboa}} & \texttt{\textbf{0.234013}} & \texttt{\textbf{}} \\
 \hline
 \texttt{\textbf{5 01/01/2020 01/01/2021}} & \vspace{-5pt}\texttt{\textbf{0.781794}} & \texttt{\textbf{}} \\
 \hline
 \texttt{\textbf{5 01/01/2021 01/01/2022}} & \vspace{-5pt}\texttt{\textbf{1.005023}} & \texttt{\textbf{}} \\
 \hline
 \texttt{\textbf{6 Faro 01/01/2020 01/01/2021}} & \vspace{-5pt}\texttt{\textbf{0.153627}} & \texttt{\textbf{}} \\
 \hline
 \texttt{\textbf{6 Braga 01/01/2021 01/02/2021}} & \vspace{-5pt}\texttt{\textbf{0.157949}} & \texttt{\textbf{}} \\
 \hline
 \texttt{\textbf{7 10 Porto}} & \texttt{\textbf{0.588885}} & \texttt{\textbf{}} \\
 \hline
 \texttt{\textbf{8 M 12}} & \texttt{\textbf{0.176378}} & \texttt{\textbf{}} \\
 \hline
 \texttt{\textbf{8 F 12}} & \texttt{\textbf{0.184072}} & \texttt{\textbf{}} \\
 \hline
 \texttt{\textbf{9 24/12/2021 25/12/2021}} & \vspace{-5pt}\texttt{\textbf{0.454373}} & \texttt{\textbf{}} \\
 \hline
 \texttt{\textbf{9 9 01/01/2012 01/01/2013}} & \vspace{-5pt}\texttt{\textbf{0.420670}} & \texttt{\textbf{}} \\
 \hline
 \end{tabularx}
}


\section{Leitura e Validação dos Dataset}

{
\setlength\arrayrulewidth{1pt}

\begin{center}
    \begin{tabular}{ |p{4.73cm}|p{4.73cm}|p{4.73cm}|  }
        \hline
        \multicolumn{3}{|c|}{\textbf{\texttt{Dataset Regular (s)}}} \\
        \hline
        \centering\textbf{\texttt{Utilizadores}} & \hfil \textbf{\texttt{Condutores}} & \hfil \textbf{\texttt{Viagens/Cidades}} \\
        \hline
        \centering\textbf{\texttt{0.074244}} & \hfil \texttt{\textbf{0.008443}} & \hfil \texttt{\textbf{0.882962}}\\
        \hline
    \end{tabular}
\end{center}


\begin{center}
    \begin{tabular}{ |p{4.73cm}|p{4.73cm}|p{4.73cm}|  }
        \hline
        \multicolumn{3}{|c|}{\textbf{\texttt{Dataset Estendido (s)}}} \\
        \hline
        \centering\textbf{\texttt{Utilizadores}} & \hfil \textbf{\texttt{Condutores}} & \hfil \textbf{\texttt{Viagens/Cidades}} \\
        \hline
        \centering\textbf{\texttt{0.708699}} & \hfil \texttt{\textbf{0.083938}} & \hfil \texttt{\textbf{11.151530}}\\
        \hline
    \end{tabular}
\end{center}
}

\vspace{10pt}

Tal como a tabela evidencia, a recolha dos dados relativos aos utilizadores e condutores parece ser feita em tempo linear, ao passo que a recolha das viagens/cidades é um pouco mais complexa, sendo muito possivelmente polinomial, com um grau pouco superior a um, algo que se deve ao facto de os dados estarem a ser armazenados em diversas estruturas simultaneamente. 


\section{Gestão de Memória}

{
\setlength\arrayrulewidth{1pt}

\begin{center}
    \begin{tabular}{ |p{4.73cm}|p{4.73cm}|p{4.73cm}|  }
        \hline
        \multicolumn{3}{|c|}{\textbf{\texttt{Dataset Regular \href{https://li3.di.uminho.pt/test/161/input}{(input)}}}} \\
        \hline
        \centering\textbf{\texttt{Memória Alocada}} & \hfil \textbf{\texttt{Leak}} & \hfil \textbf{\texttt{Peak}} \\
        \hline
        \centering\textbf{\texttt{345.2 MB}} & \hfil \texttt{\textbf{0 MB}} & \hfil \texttt{\textbf{226.5 MB}}\\
        \hline
    \end{tabular}
\end{center}


\begin{center}
    \begin{tabular}{ |p{4.73cm}|p{4.73cm}|p{4.73cm}|  }
        \hline
        \multicolumn{3}{|c|}{\textbf{\texttt{Dataset Estendido \href{https://li3.di.uminho.pt/test/161/input}{(input)}}}} \\
        \hline
        \centering\textbf{\texttt{Memória Alocada}} & \hfil \textbf{\texttt{Leak}} & \hfil \textbf{\texttt{Peak}} \\
        \hline
        \centering\textbf{\texttt{3240.4 MB}} & \hfil \texttt{\textbf{0 MB}} & \hfil \texttt{\textbf{2126.2 MB}}\\
        \hline
    \end{tabular}
\end{center}
}

\vspace{10pt}

A partir destes dados percebemos que a quantidade de memória exigida pelo programa cresce de forma linear, algo que faz todo o sentido, uma vez que a quantidade de dados que devem ser guardados também segue essa tendência, além disso, notamos que o programa revela um determinado nível de segurança relativamente à gestão de memória, dado que não foram detetados quaisquer \textit{memory leaks}.  

\chapter{Análise de complexidade}

Para melhor compreendermos o porquê de o tempo requerido em algumas \textit{queries} ser tão distinto do de outras, pensamos que a melhor forma de explicar tais valores seja recorrer a uma análise assimptótica, na qual verificamos a complexidade de cada umas das \textit{queries} tendo em conta diversos parâmetros ao invés de fixarmos um valor em especifico, como por exemplo o número de viagens realizadas. 

\section{Query 1}

Esta \textit{query} possui uma complexidade de \( N \in \theta(N)\), sendo que \(N\) representa o número de viagens realizadas por um utilizador/condutor, deste modo, poderíamos pensar que a análise de um condutor levaria mais tempo, já que normalmente este possui mais viagens, contudo como os utilizadores estão contidos numa tabela de \textit{hash} (onde cada \textit{bucket} é na verdade uma lista ligada), daí resulta que o acesso a estes é bastante custoso e moroso, visto que estão dispersos pela memória.

Assim sendo, e tal como verificamos pelas múltiplas execuções desta \textit{query}, um condutor necessita de menos tempo que um utilizador para ser analisado.

\section{Query 2}

Nesta análise é pedida uma listagem dos \(N\) condutores com maior avaliação média, portanto, para resolver este problema, tivemos de identificar cada um dos condutores e analisar as suas respetivas viagens, pelo que se pode dizer que no fundo percorremos o \textit{array} das viagens de cima a baixo.

Depois disso, foi ainda necessário ordenar a estrutura de dados resultante da recolha inicial, operação que foi feita a partir do famoso algoritmo de ordenação \textit{quick sort}.

Posto isto, a complexidade desta \textit{query} corresponde a \(N + P + P\log P \in\theta(N)\), sendo que \(N\) corresponde ao número de viagens e \(P\) ao número de condutores, pelo que se percebe que o tempo de execução não depende do parâmetro \textit{top n} passado com argumento.

\section{Query 3}

Esta \textit{query} é muito semelhante à anterior, muda o facto de se pretender listar os \(N\) utilizadores com maior distância viajada, assim sendo, se na análise anterior foi necessário percorrer o \textit{array} dos condutores por completo, neste caso teremos de percorrer a tabela de \textit{hash} onde os utilizadores estão presentes, o que não é tão eficiente dado que esta estrutura não beneficia da mesma localidade espacial que os \textit{arrays}.

Deste modo, a complexidade equivale a \(N + S + S\log S\in\theta(N)\), onde \(S\) corresponde ao número de utilizadores, o que é bastante credível, uma vez que o tempo de execução é bastante próximo do da análise feita anteriormente.

\section{Query 4}

Neste problema é pedido para calcular o preço médio das viagens numa determinada cidade, como tal, a melhor forma de abordar esta questão é tirar partido da estrutura de dados das viagens e das cidades, de modo a obter todos os índices cujas viagens foram realizadas na cidade em questão.

Depois de obter todos os índices, basta percorrer essas mesmas posições e ir acumulado o preço de cada uma das viagens, de modo a fazer uma média ponderada no final. Assim, a complexidade desta \textit{query} corresponde a \(N/C \in\theta(N/C)\), sendo que \(C\) representa o número de cidades existentes, isto assumindo que as viagens estão igualmente distribuídas pelas cidades.

\section{Query 5}

Esta \textit{query} é uma das poucas em que não fomos capazes que criar uma estrutura auxiliar de modo a obter uma execução mais rápida, como tal, fomos obrigados a percorrer o \textit{array} das viagens sem que no fundo houvesse essa necessidade, visto que tivemos de analisar viagens que nem sequer respeitavam os parâmetros pedidos pela \textit{query}.

Assim, a complexidade desta \textit{query} equivale a \(N \in \theta(N)\), o que não reflete lá muito bem os tempos obtidos, dado que esta análise beneficia bastante da localidade espacial.

\section{Query 6}

A exemplo do problema anterior, também neste caso não fomos capazes de realizar qualquer otimização referente às datas, portanto, apenas diminuímos o número de iterações realizadas graças ao parâmetro \textit{cidade}, dado que a partir do \textit{array} das cidades sabemos \textit{a priori} quais os índices das viagens em que a cidade em questão ocorre. 

Deste modo, como temos de analisar todas as viagens realizadas numa determinada cidade, concluímos que a complexidade desta \textit{query} é idêntica à da \textit{query} 4, ou seja \(N/C \in \theta(N/C)\), o que faz sentido, umas vez que os resultados obtidos em ambas as análise são bastante idênticos.

\section{Query 7}

Mais uma vez, o parâmetro \textit{cidade} é um dos pontos chave no que à execução de \textit{queries} diz respeito, assim sendo, apenas necessitamos de analisar determinadas posições da estrutura de dados das viagens e ir adicionando os condutores numa árvore binária de procura de modo a calcular a avaliação média de todos os condutores.

Depois de calcular todos os valores necessários, basta copiar os elementos presentes na árvore binária para um \textit{array}, para que assim possamos fazer uma ordenação a fim de obter o \textit{top n} mais facilmente.

Posto isto, a complexidade desta \textit{query} corresponde a \(\theta(N/C\log N/C)\), sendo que \(N\) equivale ao número de viagens e \(C\) ao número total de cidades identificadas, pelo que se percebe o porquê de o tempo de execução ser ligeiramente superior.

\section{Query 8}

Também neste problema foi necessário percorrer o \textit{array} das viagens, de modo a obter todos os dados necessários à realização de uma ordenação o mais correta possível, consequentemente obtemos uma complexidade de \(N + A\log A \in \theta(N)\), sendo que \(A\) equivale ao número de viagens consideradas válidas segundo os parâmetros apresentados na \textit{query}, e \(N\) ao número total de viagens realizadas.

Assim, observamos que o tempo de execução desta \textit{query} é similar ao de outras \textit{queries} que executam em tempo linear, o que a princípio pode parecer um pouco estranho, todavia faz todo o sentido, uma vez que o fator \(N\) é, normalmente, muito superior que o fator \(A\), portanto como \(N\) é linear, consequentemente a execução aproxima-se do tempo linear.

Não obstante, se o valor de \(A\) fosse próximo de \(N\), veríamos o tempo de execução crescer significativamente.

\section{Query 9}

Por fim, como esta análise possui parâmetros relativos a datas, tivemos de voltar a percorrer o \textit{array} das viagens e recolher todos os dados necessários a fim de fazer uma ordenação final, deste modo, a complexidade desta \textit{query} corresponde a \(N + N\log N \in \theta(N\log N)\). 




\section{Complexidade das Queries}

{
\setlength\arrayrulewidth{1pt}
\begin{tabularx}{\textwidth} { 
  | >{\centering\arraybackslash}X
  | >{\centering\arraybackslash}X
  | >{\centering\arraybackslash}X
  | >{\centering\arraybackslash}X | }
 \hline
 Query & Melhor Caso & Caso Médio & Pior Caso  \\
 \hline
 1 & \(\theta(1)\) & \(\theta(U)\) & \(\theta(U)\) \\
 \hline
 2 & \(\theta(N)\) & \(\theta(N)\) & \(\theta(N\log N)\) \\
 \hline
 3 & \(\theta(N)\) & \(\theta(N)\) & \(\theta(N\log N)\)  \\
 \hline
 4 & \(\theta(1)\) & \(\theta(N/C)\) & \(\theta(N)\)  \\
 \hline
 5 & \(\theta(N)\) & \(\theta(N)\) & \(\theta(N)\)  \\
 \hline
 6 & \(\theta(1)\) & \(\theta(N/C)\) & \(\theta(N)\)  \\
 \hline
 7 & \(\theta(1)\) & \(\theta(N/C)\) & \(\theta(N/C\log N/C)\)  \\
 \hline
 8 & \(\theta(N)\) & \(\theta(N)\) & \(\theta(N\log N)\)  \\
 \hline
 9 & \(\theta(N)\) & \(\theta(N\log N)\) & \(\theta(N\log N)\)  \\
 \hline
\end{tabularx}
}

\vspace{10pt}

\normalsize\textbf{Lengenda}

\begin{enumerate}
    \item \(U \rightarrow\) número de viagens realizadas por um utilizador/condutor.
    \item \(N \rightarrow\) número total de viagens consideradas válidas. 
    \item \(C \rightarrow\) número total de cidades existentes
\end{enumerate}



\end{document}
